Wie Ihr dem folgenden Stundenplan entnehmen könnt, enthält das Informatik-Studium im ersten
Semester neben der Informatikvorlesung auch eine gute Portion Mathe.

\begin{center}
\begin{tabular}{|c|c|c|c|c|c|} \hline
Zeit & Montag & Dienstag & Mittwoch & Donnerstag & Freitag \\
\hline\hline
08 -- 09  & & Informatik & & &\\
\cline{1-2}\cline{4-6}
09 -- 10  & & der Systeme& & &\\
\cline{1-2}\cline{4-6}
10 -- 11 & Mathematik II & (Menth) & Mathematik II & & \\
\cline{1-1}\cline{3-3}\cline{5-6}
11 -- 12 & (\Matheprof) & & (\Matheprof) & &\\
\hline
12 -- 13 & & & & &\\
\hline
13 -- 14 & & & & &\\
\hline
14 -- 15 & & Informatik II & & Informatik II  &\\
\cline{1-2}\cline{4-4}\cline{6-6}
15 -- 16 & & (\Infoprof) & & (\Infoprof) & \\
\hline
16 -- 17 & & & & &\\
\hline
17 -- 18 & & & & & \\
\hline
\end{tabular}
\end{center}

%Dieser Plan gilt für das erste Semester Informatik. Es kommen noch jeweils zwei Stunden
%für die Übungen zu den Vorlesungen dazu sowie Veranstaltungen aus den zwei weiteren oben genannten Bereichen.
%Die Zeiten für die Übungsgruppen
%werden innerhalb der ersten Woche in den Vorlesungen bekannt gegeben.

Dieser Plan gilt für das zweite Semester Informatik, für euch ist es jedoch das erste. Es kommen noch jeweils Übungsstunden
 zu den Vorlesungen sowie eine Veranstaltung aus dem Bereich Schlüsselqualifikation hinzu. Diese könnt ihr völlig frei aus allen Bereichen der Universität wählen, solange die Veranstaltung mit ECTS Punkten versehen und mit einer Note bewertet wird (nur Sportkurse sind nicht erlaubt!).
Die Zeiten für die Übungsgruppen
werden innerhalb der ersten Woche in den Vorlesungen bekannt gegeben.
Desweiteren wird es für die Mathe II und die Info II auch noch freiwillige Zusatz-Tutorien für euch geben, damit euch der Einstieg etwas leichter fällt.
